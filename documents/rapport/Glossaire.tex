% Dans le préambule, utiliser la commande
% 	\newglossaryentry{latex}
% 	{
% 	    name=latex,
% 	    description={Is a mark up language specially suited 
% 	    for scientific documents}
% 	}
% ou, pour un acronyme,
%  	\newacronym{ALI}{ALI}{Amplificateur Linéaire Intégré}
% 
% A utiliser avec les commandes \Gls{ALI} ou \gls{ALI} dans le texte
% Ne pas toucher à la ligne suivante, elle est placée pour informer le linter, chktex, d'ignorer un type d'erreur concernant les guillemets dans les définitions
% chktex-file 18


\newglossaryentry{fantome}
{
    name=fantôme,
    description={Un fantôme est un trajet précédent affiché sur la carte. Généralement, notre record personnel.}
}

\newglossaryentry{tag}
{
    name=tag,
    description={Un tag est une étiquette qui permet de catégoriser un trajet. Aussi appelée catégorie.}
}

\newglossaryentry{etat_relax}
{
    name={état "relax"},
description={Symbolise l'état de l'utilisateur lors de son trajet. Si celui ci est pressé de partir, il n'est pas relax. A l'inverse, si il est en avance, il ne cherchera pas à courir et sera donc considéré comme "relax"}
}

\newglossaryentry{django}
{
    name={django},
description={Framework python permettant de simplifier la conception de sites web avec le principe DRY (\textit{Don't Repeat Yourself}), et une variante de la conception MVC (Modele, Vue, Contrôleur)}
}


\newacronym{GPS}{GPS}{Global Positioning System}
\newacronym{API}{API}{Application Programming Interface}
\newacronym{REST}{REST}{REpresentational State Transfer}