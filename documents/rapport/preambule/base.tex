% Supprime des warnings lors de la compilation avec certains plugins obsolètes.
\usepackage{scrhack}

% Utilisons UTF-8
\usepackage[utf8]{inputenc}
\usepackage[T1]{fontenc}

% Traduire le document en français
\usepackage[french]{babel}

% Pour ajouter des symboles à LaTeX comme ©
% Pourrait probablement être abandonné car non maintenu
% \usepackage{textcomp}

% Pour avoir des liens fonctionnels dans le PDF
\usepackage{hyperref}
\usepackage{url}

% Pour definir et choisir des couleurs
\usepackage[svgnames, table]{xcolor}

% Active la commande \footnote
\usepackage{footnote}

% Permet d'utiliser les guillemets français avec \og et \fg
% https://tex.stackexchange.com/questions/281279/style-of-quotation-marks
\usepackage[autostyle]{csquotes}

% Permet de forcer la présence d'un flotant "ici" avec un grand H
\usepackage{float}

% Permet de placer du texte côte-à-côte avec des figures
% Utiliser avec 
% \begin{wrapfigure}{r}{5cm} FIGURE \end{wrapfigure}
% r/l/i/o désignant le côté et 5cm la taille de la figure (en majuscule plus ou moins)
% CF http://mirrors.standaloneinstaller.com/ctan/macros/latex/contrib/wrapfig/wrapfig-doc.pdf
\usepackage{wrapfig}

% Permet d'utiliser un ensemble de symboles et fonctions mathématiques
\usepackage{amsmath}
\usepackage{amssymb}

% Permet d'ajouter des images
\usepackage{graphicx}

% Permet d'inclure un PDF depuis le code LaTeX, par exemple pour une page de titre customisée (commande includepdf)
\usepackage{pdfpages}

% Permet d'écrire du texte sans sens, ce qui permet une visualisation rapide du thème et des parties pleines
\usepackage{blindtext}

% Calme LaTeX sur la justification à outrance, ce qui évite de couper un mot sur trois
\global\tolerance=2000
\global\hyphenpenalty=30000


