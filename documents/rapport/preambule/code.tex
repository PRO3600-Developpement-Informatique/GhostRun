% Paquet permettant l'affichage de blocs de code
% Deux moyens de l'utiliser, soit on importe le code depuis un fichier
% 	\lstinputlisting[language={[x86masm]Assembler}, caption=Légende, label=listing:nom_du_label, float]{FICHIER.EXT}
% Soit directement dans le code LaTeX
% 	\begin{lstlisting} et \end{lstlisting}
% Avec les memes options que précédemment.

% On utilise listings pour afficher du code
\usepackage{listings}

%New colors defined below
\definecolor{codegreen}{rgb}{0,0.6,0}
\definecolor{codegray}{rgb}{0.5,0.5,0.5}
\definecolor{codepurple}{rgb}{0.58,0,0.82}
\definecolor{backcolour}{rgb}{0.95,0.95,0.92}


\lstdefinestyle{code_style}{
  backgroundcolor=\color{backcolour},   commentstyle=\color{codegreen},
  keywordstyle=\color{magenta},
  numberstyle=\tiny\color{codegray},
  stringstyle=\color{codepurple},
  basicstyle=\ttfamily\footnotesize,
  breakatwhitespace=false,         
  breaklines=true,                 
  captionpos=b,                    
  keepspaces=true,                 
  numbers=left,                    
  numbersep=5pt,                  
  showspaces=false,                
  showstringspaces=false,
  showtabs=false,                  
  tabsize=2
}

\lstset{style=code_style}