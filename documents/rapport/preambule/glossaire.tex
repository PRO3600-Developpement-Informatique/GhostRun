% Utilisation des glossaires
%
% Dans le préambule, utiliser la commande
% 	\newglossaryentry{latex}
% 	{
% 	    name=latex,
% 	    description={Is a mark up language specially suited 
% 	    for scientific documents}
% 	}
% ou, pour un acronyme,
%  	\newacronym{ALI}{ALI}{Amplificateur Linéaire Intégré}
% 
% A utiliser avec les commandes \Gls{ALI} ou \gls{ALI} dans le texte
%
% Pour afficher le glossaire, les trois lignes suivantes font l'affaire, 
%	\appendix
%	\printglossaries
%	\glsaddallunused % < Pour ajouter les entrées non utilisées dans le texte
%
% Plus d'infos : https://www.overleaf.com/learn/latex/Glossaries

% Importation du package glossaires


\usepackage[
	style=super,
	%nonumberlist,
	toc,
	xindy,
	acronym,
	numberedsection
	]{glossaries-extra}
	
\setabbreviationstyle[acronym]{long-short}
\glssetcategoryattribute{acronym}{glossdesc}{title}
\renewcommand{\glsnamefont}[1]{\textbf{#1}}

\renewcommand*{\glstextformat}[1]{\textsf{#1}}

\makeglossaries%
