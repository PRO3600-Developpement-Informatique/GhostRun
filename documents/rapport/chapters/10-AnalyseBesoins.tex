\chapter{Analyse des besoins (Cahier des charges)}

GhostRun est une \textbf{application Android},  sur laquelle l'utilisateur se connecte pour sauvegarder des traces GPS, appelées trajets, qui s'afficheront sur la carte sous la forme d'un \gls{fantome} pendant ses futurs trajets.

Les trajets sont rangés par moyen de transport, auquel on peut y ajouter des \glspl{tag}.
Il est aussi possible de trier les trajets par données (ex : vitesse moyenne, vitesse max, temps...) et l'utilisateur peut partager ses trajets sur les réseaux sociaux.
Toutes ces données sont également accessibles via un site internet.

Avant d'enregistrer un trajet, l'utilisateur doit d'abord spécifier son itinéraire (départ et arrivée), et le moyen de transport qu'il utilisera. 

Pedant l'enregistrement, il peut spécifier quelle action il réalise. Par exemple, pour un trajet en Train, l'utilisateur passe par les modes "préparation", "marche vers la station", "achat d'un billet", "dans le train", "marche vers la destination".

Lors de l'enregistrement d'un trajet, l'utilisateur voit sur un carte l'ensemble des \glspl{fantome} qu'il a auparavant sélectionné. Il peut ainsi tente de les dépasser en temps reel. Ces \glspl{fantome} sont représentés par des icônes symbolisant le moyen de transport utilisé par l'utilisateur lorsqu'ils ont été enregistrés. L'application doit être utilisable la nuit, elle intègre donc un mode sombre, dans lequel la carte aborde un terme noir, et ou les icônes des \glspl{fantome} sont moins contrastées.

Après avoir enregistré son trajet, l'utilisateur peut y ajouter des metadonnées : conditions particulières, météo, essoufflement, \gls{etat_relax}.


Un bouton permet à l'utilisateur de partager ses scores et l'ensemble de ses \glspl{fantome} si il le désire.