\chapter{Analyse des besoins (Cahier des charges)}

GhostRun est une \textbf{application Android}, sur laquelle l'utilisateur se connecte pour sauvegarder des traces GPS, appelées trajets, qui s'afficheront sur la carte sous la forme d'un \gls{fantome} pendant ses futurs trajets.

Avant d'enregistrer un trajet, l'utilisateur doit d'abord spécifier son itinéraire (départ et arrivée), et le moyen de transport qu'il utilisera. 

Pedant l'enregistrement, il peut spécifier quelle action il réalise. Par exemple, pour un trajet en train, l'utilisateur passe par les modes "préparation", "marche vers la station", "achat d'un billet", "attente du train", "dans le train", "marche vers la destination".

Lors de l'enregistrement d'un trajet, l'utilisateur voit sur un carte l'ensemble des \glspl{fantome} qu'il a auparavant sélectionné. Il peut ainsi tenter de les dépasser en temps reel. Ces \glspl{fantome} sont représentés par des icônes symbolisant le moyen de transport utilisé par l'utilisateur lorsqu'ils ont été enregistrés. L'application doit être utilisable la nuit, elle intègre donc un mode sombre, dans lequel la carte aborde un terme noir, et ou les icônes des \glspl{fantome} sont moins contrastées.

Après avoir enregistré son trajet, l'utilisateur peut y ajouter des metadonnées : conditions particulières, météo, essoufflement, \gls{etat_relax}. L'application calcule et stocke aussi sa vitesse moyenne, maximale, la distance parcourue et d'autres statistiques.

L'application propose aussi un historique des trajets, sur lequel l'utilisateur peut décider d'afficher le trajet en tant que \gls{fantome} lors de l'enregistrement, consulter les meilleurs trajets (triés sur une multitude de critères tels que ceux énoncés précédemment), consulter ses trajets par moyen de transport (voiture, train, bus, vélo, marche, course, metro, RER, ...)

Un bouton permet à l'utilisateur de partager ses scores et l'ensemble de ses \glspl{fantome} si il le désire.

L'application synchronise en arrière plan les données générées par les trajets afin de les sauvegarder et de pouvoir les afficher sur un site web annexe.