\chapter{Conclusion}

Dans ce TP, nous avons pu utiliser deux techniques différentes de démodulation: tout d'abord la démodulation d'enveloppe, dont le circuit est simple à réaliser, mais qui demande des composants plus rares (diode au germanium) et surtout qui n'est pas capable de correctement démoduler un signal sur-modulé (quand le coefficient de modulation est supérieur à 1)

Ensuite, nous avons pu démoduler un signal par multiplication, qui nécessite un circuit légèrement plus compliqué que celui de la démodulation d'enveloppe, mais qui est capable de démoduler le signal reçu, même si celui-ci présente un coefficient de modulation supérieur à 1 (c'est à dire en sur-modulation).

Il peut donc être intéressant d'utiliser la démodulation par multiplication dans le cas où le signal (peut) présenter une surmodulation.