\chapter{Manuel Utilisateur}

\section{Application Android}

L'application Android permet à l'utilisateur de suivre son trajet sur une carte Google Maps et de faire une course contre un autre trajet.

Tout d'abord, lors du lancement de l'application, il faut se connecter avec son identifiant et son mot de passe pour arriver sur la carte. La navigation se fait avec trois onglets. Le premier onglet correspond à la carte où l'on peut voir sa position et de plus le trajet que l'on est en train de faire durant une course.

Le deuxième onglet montre toutes les catégories que l'utilisateur a créé et il peut en créer de nouvelles. Dans ces catégories, l'utilisateur peut choisir le trajet contre lequel il veut faire la course.

Et enfin le troisième onglet correspond à la page des options où l'utilisateur peut par exemple changer le thème de la carte en plus sombre (une actualisation de la position sur la carte est nécessaire pour que le thème change) ou encore se déconnecter.

\section{Site web}

Le site web a une interface très simple d'utilisation. La page d'accueil décrit très simplement le but de l'application GhostRun.

En haut à gauche de la page d'accueil, nous avons deux boutons, Espace membres et préférences, qui ne sont disponibles que si l'utilisateur s'est connecté. Il peut le faire avec le bouton Connexion en haut à droite. Les identifiants de l'utilisateur sont les mêmes que ceux sur l'application.

Après s'être connecté, l'utilisateur a donc accès à son espace membres où il peut voir tous les trajets qu'il a réalisé dans chacune de ses catégories. En cliquant sur l'un des trajets, il peut voir plusieurs statistiques sur ce trajet (comme la distance parcourue, sa vitesse moyenne ou encore la durée du trajet). Tout ceci peut lui permettre de faire son choix sur quel trajet il souhaite réaliser une course.

Et enfin, la page des préférences où (A compléter...)

\section{Interface administrateur}


\section{API}